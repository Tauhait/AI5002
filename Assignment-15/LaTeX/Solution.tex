\documentclass[journal,12pt,twocolumn]{IEEEtran}
\usepackage{setspace}
\usepackage{gensymb}
\singlespacing
\usepackage[cmex10]{amsmath}

\usepackage{amsthm}
\usepackage{hyperref}
\hypersetup{
    colorlinks=true,
    linkcolor=blue,
    filecolor=magenta,      
    urlcolor=cyan,
}

\urlstyle{same}
\usepackage{mathrsfs}
\usepackage{txfonts}
\usepackage{stfloats}
\usepackage{bm}
\usepackage{cite}
\usepackage{cases}
\usepackage{subfig}

\usepackage{longtable}
\usepackage{multirow}

\usepackage{enumitem}
\usepackage{mathtools}
\usepackage{steinmetz}
\usepackage{tikz}
\usepackage{circuitikz}
\usepackage{verbatim}
\usepackage{tfrupee}
\usepackage[breaklinks=true]{hyperref}
\usepackage{graphicx}
\usepackage{tkz-euclide}
\usetikzlibrary{shapes,backgrounds}
\usepackage{verbatim}
\usetikzlibrary{calc,math}
\usepackage{listings}
    \usepackage{color}                                            %%
    \usepackage{array}                                            %%
    \usepackage{longtable}                                        %%
    \usepackage{calc}                                             %%
    \usepackage{multirow}                                         %%
    \usepackage{hhline}                                           %%
    \usepackage{ifthen}                                           %%
    \usepackage{lscape}     
\usepackage{multicol}
\usepackage{chngcntr}
\usepackage{mdframed}
\DeclareMathOperator*{\Res}{Res}

\renewcommand\thesection{\arabic{section}}
\renewcommand\thesubsection{\thesection.\arabic{subsection}}
\renewcommand\thesubsubsection{\thesubsection.\arabic{subsubsection}}

\renewcommand\thesectiondis{\arabic{section}}
\renewcommand\thesubsectiondis{\thesectiondis.\arabic{subsection}}
\renewcommand\thesubsubsectiondis{\thesubsectiondis.\arabic{subsubsection}}


\hyphenation{op-tical net-works semi-conduc-tor}
\def\inputGnumericTable{}                                 %%

\lstset{
%language=C,
frame=single, 
breaklines=true,
columns=fullflexible
}

\usepackage{chngcntr}
\counterwithin{figure}{section}

\title{AI5002}
\author{TUHIN DUTTA}
\date{January 2021}

\begin{document}
\newtheorem{theorem}{Theorem}[section]
\newtheorem{problem}{Problem}
\newtheorem{proposition}{Proposition}[section]
\newtheorem{lemma}{Lemma}[section]
\newtheorem{corollary}[theorem]{Corollary}
\newtheorem{example}{Example}[section]
\newtheorem{definition}[problem]{Definition}

\newcommand{\BEQA}{\begin{eqnarray}}
\newcommand{\EEQA}{\end{eqnarray}}
\newcommand{\define}{\stackrel{\triangle}{=}}
\bibliographystyle{IEEEtran}
\raggedbottom
\setlength{\parindent}{0pt}
\providecommand{\mbf}{\mathbf}
\providecommand{\pr}[1]{\ensuremath{\Pr\left(#1\right)}}
\providecommand{\qfunc}[1]{\ensuremath{Q\left(#1\right)}}
\providecommand{\sbrak}[1]{\ensuremath{{}\left[#1\right]}}
\providecommand{\lsbrak}[1]{\ensuremath{{}\left[#1\right.}}
\providecommand{\rsbrak}[1]{\ensuremath{{}\left.#1\right]}}
\providecommand{\brak}[1]{\ensuremath{\left(#1\right)}}
\providecommand{\lbrak}[1]{\ensuremath{\left(#1\right.}}
\providecommand{\rbrak}[1]{\ensuremath{\left.#1\right)}}
\providecommand{\cbrak}[1]{\ensuremath{\left\{#1\right\}}}
\providecommand{\lcbrak}[1]{\ensuremath{\left\{#1\right.}}
\providecommand{\rcbrak}[1]{\ensuremath{\left.#1\right\}}}
\theoremstyle{remark}
\newtheorem{rem}{Remark}
\newcommand{\sgn}{\mathop{\mathrm{sgn}}}

\providecommand{\res}[1]{\Res\displaylimits_{#1}} 

%\providecommand{\norm}[1]{\lVert#1\rVert}
\providecommand{\mtx}[1]{\mathbf{#1}}
\providecommand{\fourier}{\overset{\mathcal{F}}{ \rightleftharpoons}}
%\providecommand{\hilbert}{\overset{\mathcal{H}}{ \rightleftharpoons}}
\providecommand{\system}{\overset{\mathcal{H}}{ \longleftrightarrow}}
	%\newcommand{\solution}[2]{\textbf{Solution:}{#1}}
\newcommand{\solution}{\noindent \textbf{Solution: }}
\newcommand{\cosec}{\,\text{cosec}\,}
\providecommand{\dec}[2]{\ensuremath{\overset{#1}{\underset{#2}{\gtrless}}}}
\newcommand{\myvec}[1]{\ensuremath{\begin{pmatrix}#1\end{pmatrix}}}
\newcommand{\mydet}[1]{\ensuremath{\begin{vmatrix}#1\end{vmatrix}}}
\numberwithin{equation}{subsection}
\makeatletter
\@addtoreset{figure}{problem}
\makeatother
\let\StandardTheFigure\thefigure
\let\vec\mathbf
\renewcommand{\thefigure}{\theproblem}
\def\putbox#1#2#3{\makebox[0in][l]{\makebox[#1][l]{}\raisebox{\baselineskip}[0in][0in]{\raisebox{#2}[0in][0in]{#3}}}}
     \def\rightbox#1{\makebox[0in][r]{#1}}
     \def\centbox#1{\makebox[0in]{#1}}
     \def\topbox#1{\raisebox{-\baselineskip}[0in][0in]{#1}}
     \def\midbox#1{\raisebox{-0.5\baselineskip}[0in][0in]{#1}}
\vspace{3cm}
\title{AI5002 - Assignment 15}
\author{Tuhin Dutta\\ ai21mtech02002}
\maketitle
\newpage
\bigskip
\renewcommand{\thefigure}{\theenumi}
\renewcommand{\thetable}{\theenumi}
\begin{mdframed}
Download code and LaTeX from below hyperlinks\\
1. \href{https://github.com/Tauhait/AI5002/tree/main/Assignment-15/LaTeX}{LaTeX}
\end{mdframed}
\subsection*{\boldsymbol{Problem\ \textbf{NET\_JUNE\_2012\_Q107}}}
Let $X_1$, $X_2$, ... be independent random variables with\\ $X_n\ \sim$ U(-n, 3n) where n = 1, 2, ...\\

Let $S_N = \dfrac{1}{\sqrt{N}}\ \sum_{n=1}^{N}\ \dfrac{X_n}{n}\ for\ N = 1, 2, ..\ \infty$\\

Let $F_N$ be the distribution function of $S_N$\\
Also let $\phi$ denote the distribution function of a standard normal random variable. Which of the following is/are true?\\
\begin{enumerate}
    \item \tag{1} $\lim_{N \to \infty}\ F_N(0) \leq\ \phi\ (0)$
    \item \tag{2} $\lim_{N \to \infty}\ F_N(0) \geq\ \phi\ (0)$
    \item \tag{3} $\lim_{N \to \infty}\ F_N(1) \leq\ \phi\ (1)$
    \item \tag{4} $\lim_{N \to \infty}\ F_N(1) \geq\ \phi\ (1)$
\end{enumerate}
\subsection*{\boldsymbol{Solution}}
Let 
\begin{equation}\tag{1}
    Y_n = \dfrac{X_n - n}{3n}
\end{equation}
And by a simple computation we can see,
\begin{equation}\tag{2}
    Y_n\ is\ i.i.d.\ \sim U(-\dfrac{2}{3},\ \dfrac{2}{3})
\end{equation}
Also from (1), we can write 
\begin{equation}\tag{3}
    X_n = 3nY_n + n
\end{equation}
Now,
\begin{equation}\tag{4}
    \begin{split}
        S_N &= \dfrac{1}{\sqrt{N}}\ \sum_{n=1}^{N}\ \dfrac{X_n}{n}\\
            &From\ (3)\ we\ can\ write,\\
            &= \dfrac{1}{\sqrt{N}}\ \sum_{n=1}^{N}\ \dfrac{n + 3nY_n}{n}\\
            &= \dfrac{1}{\sqrt{N}}\ \sum_{n=1}^{N}\ 1 + 3Y_n\\
            &= 3\dfrac{1}{\sqrt{N}}\ \sum_{n=1}^{N}\ Y_n\ +\ \sqrt{N}\\
    \end{split}
\end{equation}
By Central Limit Theorem, the first term tends to a normal distribution and the whole R.H.S. tends to,\\
\begin{equation}\tag{5}
    \begin{split}
        &As\ (N\ \rightarrow\ \infty)\\
        &S_N = 3\dfrac{1}{\sqrt{N}}\ \sum_{n=1}^{N}\ Y_n\ +\ \sqrt{N}\ \longrightarrow\ \infty\\
    \end{split}
\end{equation}
$\because\ S_N\ \rightarrow\ \infty$, it implies  
\begin{equation}\tag{6}
    \begin{split}
        \lim_{N \to \infty}\ F_N(0) = 0\\
        \lim_{N \to \infty}\ F_N(1) = 0
    \end{split}
\end{equation}

By definition we know, a standard normal random variable is a normally distributed random variable with mean $\mu$ = 0  and standard deviation $\sigma$ = 1 . It is denoted by the letter  Z. From Z-distribution table we know,
\begin{equation}\tag{7}
    \begin{split}
        P(Z\leqslant 0) &= \phi(0) = 0.5\\
        P(Z\leqslant 1) &= \phi(1) = 0.84134
    \end{split}
\end{equation}
Hence by comparing (6) and (7) we see only options 1) and 3) are true. 
\end{document}
