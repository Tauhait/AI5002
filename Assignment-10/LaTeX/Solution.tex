\documentclass[journal,12pt,twocolumn]{IEEEtran}
\usepackage{setspace}
\usepackage{gensymb}
\singlespacing
\usepackage[cmex10]{amsmath}

\usepackage{amsthm}
\usepackage{hyperref}
\hypersetup{
    colorlinks=true,
    linkcolor=blue,
    filecolor=magenta,      
    urlcolor=cyan,
}

\urlstyle{same}
\usepackage{mathrsfs}
\usepackage{txfonts}
\usepackage{stfloats}
\usepackage{bm}
\usepackage{cite}
\usepackage{cases}
\usepackage{subfig}

\usepackage{longtable}
\usepackage{multirow}

\usepackage{enumitem}
\usepackage{mathtools}
\usepackage{steinmetz}
\usepackage{tikz}
\usepackage{circuitikz}
\usepackage{verbatim}
\usepackage{tfrupee}
\usepackage[breaklinks=true]{hyperref}
\usepackage{graphicx}
\usepackage{tkz-euclide}
\usetikzlibrary{shapes,backgrounds}
\usepackage{verbatim}
\usetikzlibrary{calc,math}
\usepackage{listings}
    \usepackage{color}                                            %%
    \usepackage{array}                                            %%
    \usepackage{longtable}                                        %%
    \usepackage{calc}                                             %%
    \usepackage{multirow}                                         %%
    \usepackage{hhline}                                           %%
    \usepackage{ifthen}                                           %%
    \usepackage{lscape}     
\usepackage{multicol}
\usepackage{chngcntr}
\usepackage{mdframed}
\DeclareMathOperator*{\Res}{Res}

\renewcommand\thesection{\arabic{section}}
\renewcommand\thesubsection{\thesection.\arabic{subsection}}
\renewcommand\thesubsubsection{\thesubsection.\arabic{subsubsection}}

\renewcommand\thesectiondis{\arabic{section}}
\renewcommand\thesubsectiondis{\thesectiondis.\arabic{subsection}}
\renewcommand\thesubsubsectiondis{\thesubsectiondis.\arabic{subsubsection}}


\hyphenation{op-tical net-works semi-conduc-tor}
\def\inputGnumericTable{}                                 %%

\lstset{
%language=C,
frame=single, 
breaklines=true,
columns=fullflexible
}

\usepackage{chngcntr}
\counterwithin{figure}{section}

\title{AI5002}
\author{TUHIN DUTTA}
\date{January 2021}

\begin{document}
\newtheorem{theorem}{Theorem}[section]
\newtheorem{problem}{Problem}
\newtheorem{proposition}{Proposition}[section]
\newtheorem{lemma}{Lemma}[section]
\newtheorem{corollary}[theorem]{Corollary}
\newtheorem{example}{Example}[section]
\newtheorem{definition}[problem]{Definition}

\newcommand{\BEQA}{\begin{eqnarray}}
\newcommand{\EEQA}{\end{eqnarray}}
\newcommand{\define}{\stackrel{\triangle}{=}}
\bibliographystyle{IEEEtran}
\raggedbottom
\setlength{\parindent}{0pt}
\providecommand{\mbf}{\mathbf}
\providecommand{\pr}[1]{\ensuremath{\Pr\left(#1\right)}}
\providecommand{\qfunc}[1]{\ensuremath{Q\left(#1\right)}}
\providecommand{\sbrak}[1]{\ensuremath{{}\left[#1\right]}}
\providecommand{\lsbrak}[1]{\ensuremath{{}\left[#1\right.}}
\providecommand{\rsbrak}[1]{\ensuremath{{}\left.#1\right]}}
\providecommand{\brak}[1]{\ensuremath{\left(#1\right)}}
\providecommand{\lbrak}[1]{\ensuremath{\left(#1\right.}}
\providecommand{\rbrak}[1]{\ensuremath{\left.#1\right)}}
\providecommand{\cbrak}[1]{\ensuremath{\left\{#1\right\}}}
\providecommand{\lcbrak}[1]{\ensuremath{\left\{#1\right.}}
\providecommand{\rcbrak}[1]{\ensuremath{\left.#1\right\}}}
\theoremstyle{remark}
\newtheorem{rem}{Remark}
\newcommand{\sgn}{\mathop{\mathrm{sgn}}}

\providecommand{\res}[1]{\Res\displaylimits_{#1}} 

%\providecommand{\norm}[1]{\lVert#1\rVert}
\providecommand{\mtx}[1]{\mathbf{#1}}
\providecommand{\fourier}{\overset{\mathcal{F}}{ \rightleftharpoons}}
%\providecommand{\hilbert}{\overset{\mathcal{H}}{ \rightleftharpoons}}
\providecommand{\system}{\overset{\mathcal{H}}{ \longleftrightarrow}}
	%\newcommand{\solution}[2]{\textbf{Solution:}{#1}}
\newcommand{\solution}{\noindent \textbf{Solution: }}
\newcommand{\cosec}{\,\text{cosec}\,}
\providecommand{\dec}[2]{\ensuremath{\overset{#1}{\underset{#2}{\gtrless}}}}
\newcommand{\myvec}[1]{\ensuremath{\begin{pmatrix}#1\end{pmatrix}}}
\newcommand{\mydet}[1]{\ensuremath{\begin{vmatrix}#1\end{vmatrix}}}
\numberwithin{equation}{subsection}
\makeatletter
\@addtoreset{figure}{problem}
\makeatother
\let\StandardTheFigure\thefigure
\let\vec\mathbf
\renewcommand{\thefigure}{\theproblem}
\def\putbox#1#2#3{\makebox[0in][l]{\makebox[#1][l]{}\raisebox{\baselineskip}[0in][0in]{\raisebox{#2}[0in][0in]{#3}}}}
     \def\rightbox#1{\makebox[0in][r]{#1}}
     \def\centbox#1{\makebox[0in]{#1}}
     \def\topbox#1{\raisebox{-\baselineskip}[0in][0in]{#1}}
     \def\midbox#1{\raisebox{-0.5\baselineskip}[0in][0in]{#1}}
\vspace{3cm}
\title{AI5002 - Assignment 10}
\author{Tuhin Dutta\\ ai21mtech02002}
\maketitle
\newpage
\bigskip
\renewcommand{\thefigure}{\theenumi}
\renewcommand{\thetable}{\theenumi}
\begin{mdframed}
Download code and LaTeX from below hyperlinks\\
1. \href{https://github.com/Tauhait/AI5002/blob/main/Assignment-10/Code/GATE\_11.py}{Code/GATE\_11.py}


2. \href{https://github.com/Tauhait/AI5002/tree/main/Assignment-10/LaTeX}{LaTeX}
\end{mdframed}
\subsection*{\boldsymbol{Problem\ GATE11}}
The probability that a given positive integer
lying between 1 and 100 (both inclusive) is
NOT divisible by 2, 3 or 5 is.....

\subsection*{\boldsymbol{Solution}}\\
Number of positive integers lying between 1 and 100 (both inclusive) is given by the set S.\\n(S) = 100\\
\\
Let us define a random variable A as `Integers between 1 and 100 (both inclusive) divisible by 2'. The sample space defined by A is given by - \\
$A = \{\\
2, 4, 6, 8, 10, 12, 14, 16, 18, 20, 22, 24, 26, 28,\\
30, 32, 34, 36, 38, 40, 42, 44, 46, 48, 50, 52, 54,\\
56, 58, 60, 62, 64, 66, 68, 70, 72, 74, 76, 78, 80,\\
82, 84, 86, 88, 90, 92, 94, 96, 98, 100 \} $\\
\\
\begin{align}
    Pr(A=x) =
    \begin{cases}
        \frac{1}{50} & x \in A \\
        0 & otherwise
    \end{cases}
\end{align}
\\
Let us define a random variable B as `Integers between 1 and 100 (both inclusive) divisible by 3'. The sample space defined by B is given by - \\
$B = \{\\
3, 6, 9, 12, 15, 18, 21, 24, 27, 30, 33, 36, 39, 42,\\
45, 48, 51, 54, 57, 60, 63, 66, 69, 72, 75, 78, 81, \\
84, 87, 90, 93, 96, 99 \} $\\
\begin{align}
    Pr(B=x) =
    \begin{cases}
        \frac{1}{33} & x \in B \\
        0 & otherwise
    \end{cases}
\end{align}
\\
Let us define a random variable C as `Integers between 1 and 100 (both inclusive) divisible by 5'. The sample space defined by C is given by - \\
$C = \{\\
5, 10, 15, 20, 25, 30, 35, 40, 45, 50, 55, 60, 65, 70, \\
75, 80, 85, 90, 95, 100 \} $\\
\begin{align}
    Pr(C=x) =
    \begin{cases}
        \frac{1}{20} & x \in C \\
        0 & otherwise
    \end{cases}
\end{align}
\\
Let us define a random variable AC as `Integers between 1 and 100 (both inclusive) divisible by 2 and 5 or 10'. The sample space defined by AC is given by - \\
The sample space is given by - \\
$AC = \{ 10, 20, 30, 40, 50, 60, 70, 80, 90, 100 \} $\\
\begin{align}
    Pr(AC=x) =
    \begin{cases}
        \frac{1}{10} & x \in AC \\
        0 & otherwise
    \end{cases}
\end{align}
\\
Let us define a random variable AB as `Integers between 1 and 100 (both inclusive) divisible by 2 and 3 or 6'. The sample space defined by AB is given by - \\
$AB = \{\\
6, 12, 18, 24, 30, 36, 42, 48, 54, 60, 66, 72, 78,\\
84, 90, 96 \} $\\
\begin{align}
    Pr(AB=x) =
    \begin{cases}
        \frac{1}{16} & x \in AB \\
        0 & otherwise
    \end{cases}
\end{align}
\\
Let us define a random variable BC as `Integers between 1 and 100 (both inclusive) divisible by 3 and 5 or 15'. The sample space defined by BC is given by - \\
$BC = \{ 15, 30, 45, 60, 75, 90 \} $\\
\begin{align}
    Pr(BC=x) =
    \begin{cases}
        \frac{1}{6} & x \in BC \\
        0 & otherwise
    \end{cases}
\end{align}
\\
Let us define a random variable ABC as `Integers between 1 and 100 (both inclusive) divisible by 2, 3 and 5 or 30'. The sample space defined by ABC is given by - \\
$ABC = \{ 30, 60, 90\}$\\
\begin{align}
    Pr(ABC=x) =
    \begin{cases}
        \frac{1}{3} & x \in ABC \\
        0 & otherwise
    \end{cases}
\end{align}
\\
From inclusion-exclusion principle we know, \\
\begin{multline}
    \pr{A + B + C} = \pr{A} + \pr{B} + \pr{C}\\ - \pr{AB} - \pr{AC} - \pr{BC} + \pr{ABC}\\
\end{multline}
\begin{multline}
    \pr{A + B + C} = \dfrac{50}{100} + \dfrac{33}{100} + \dfrac{20}{100} -\\ \dfrac{16}{100} - \dfrac{10}{100} - \dfrac{6}{100} + \dfrac{3}{100} = \dfrac{74}{100} = 0.74\\
\end{multline}
%The sample space is given by -\\
% = \{ \\
% 2, 3, 4, 5, 6, 8, 9, 10, 12, 14, 15, 16, 18,\\
% 20, 21, 22, 24, 25, 26, 27, 28, 30, 32, 33,\\
% 34, 35, 36, 38, 39, 40, 42, 44, 45, 46, 48,\\
% 50, 51, 52, 54, 55, 56, 57, 58, 60, 62, 63,\\
% 64, 65, 66, 68, 69, 70, 72, 74, 75, 76, 78,\\
% 80, 81, 82, 84, 85, 86, 87, 88, 90, 92, 93,\\
% 94, 95, 96, 98, 99, 100 \}\\
\\
Let us define a random variable X as `Integers between 1 and 100 (both inclusive) NOT divisible by 2, 3, or 5'.\\
\\
The sample space is given by -\\
= \{\\
1, 7, 11, 13, 17, 19, 23, 29, 31, 37, 41, 43, 47, 49,\\
53, 59, 61, 67, 71, 73, 77, 79, 83, 89, 91, 97\\
\}\\
\\
The probability of X is given by - \\
\begin{multline}
    \pr{X} = 1 - \pr{A + B + C} = 1 - 0.74 = 0.26\\
\end{multline}
\end{document}
