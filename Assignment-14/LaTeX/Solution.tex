\documentclass[journal,12pt,twocolumn]{IEEEtran}
\usepackage{setspace}
\usepackage{gensymb}
\singlespacing
\usepackage[cmex10]{amsmath}

\usepackage{amsthm}
\usepackage{hyperref}
\hypersetup{
    colorlinks=true,
    linkcolor=blue,
    filecolor=magenta,      
    urlcolor=cyan,
}

\urlstyle{same}
\usepackage{mathrsfs}
\usepackage{txfonts}
\usepackage{stfloats}
\usepackage{bm}
\usepackage{cite}
\usepackage{cases}
\usepackage{subfig}

\usepackage{longtable}
\usepackage{multirow}

\usepackage{enumitem}
\usepackage{mathtools}
\usepackage{steinmetz}
\usepackage{tikz}
\usepackage{circuitikz}
\usepackage{verbatim}
\usepackage{tfrupee}
\usepackage[breaklinks=true]{hyperref}
\usepackage{graphicx}
\usepackage{tkz-euclide}
\usetikzlibrary{shapes,backgrounds}
\usepackage{verbatim}
\usetikzlibrary{calc,math}
\usepackage{listings}
    \usepackage{color}                                            %%
    \usepackage{array}                                            %%
    \usepackage{longtable}                                        %%
    \usepackage{calc}                                             %%
    \usepackage{multirow}                                         %%
    \usepackage{hhline}                                           %%
    \usepackage{ifthen}                                           %%
    \usepackage{lscape}     
\usepackage{multicol}
\usepackage{chngcntr}
\usepackage{mdframed}
\DeclareMathOperator*{\Res}{Res}

\renewcommand\thesection{\arabic{section}}
\renewcommand\thesubsection{\thesection.\arabic{subsection}}
\renewcommand\thesubsubsection{\thesubsection.\arabic{subsubsection}}

\renewcommand\thesectiondis{\arabic{section}}
\renewcommand\thesubsectiondis{\thesectiondis.\arabic{subsection}}
\renewcommand\thesubsubsectiondis{\thesubsectiondis.\arabic{subsubsection}}


\hyphenation{op-tical net-works semi-conduc-tor}
\def\inputGnumericTable{}                                 %%

\lstset{
%language=C,
frame=single, 
breaklines=true,
columns=fullflexible
}

\usepackage{chngcntr}
\counterwithin{figure}{section}

\title{AI5002}
\author{TUHIN DUTTA}
\date{January 2021}

\begin{document}
\newtheorem{theorem}{Theorem}[section]
\newtheorem{problem}{Problem}
\newtheorem{proposition}{Proposition}[section]
\newtheorem{lemma}{Lemma}[section]
\newtheorem{corollary}[theorem]{Corollary}
\newtheorem{example}{Example}[section]
\newtheorem{definition}[problem]{Definition}

\newcommand{\BEQA}{\begin{eqnarray}}
\newcommand{\EEQA}{\end{eqnarray}}
\newcommand{\define}{\stackrel{\triangle}{=}}
\bibliographystyle{IEEEtran}
\raggedbottom
\setlength{\parindent}{0pt}
\providecommand{\mbf}{\mathbf}
\providecommand{\pr}[1]{\ensuremath{\Pr\left(#1\right)}}
\providecommand{\qfunc}[1]{\ensuremath{Q\left(#1\right)}}
\providecommand{\sbrak}[1]{\ensuremath{{}\left[#1\right]}}
\providecommand{\lsbrak}[1]{\ensuremath{{}\left[#1\right.}}
\providecommand{\rsbrak}[1]{\ensuremath{{}\left.#1\right]}}
\providecommand{\brak}[1]{\ensuremath{\left(#1\right)}}
\providecommand{\lbrak}[1]{\ensuremath{\left(#1\right.}}
\providecommand{\rbrak}[1]{\ensuremath{\left.#1\right)}}
\providecommand{\cbrak}[1]{\ensuremath{\left\{#1\right\}}}
\providecommand{\lcbrak}[1]{\ensuremath{\left\{#1\right.}}
\providecommand{\rcbrak}[1]{\ensuremath{\left.#1\right\}}}
\theoremstyle{remark}
\newtheorem{rem}{Remark}
\newcommand{\sgn}{\mathop{\mathrm{sgn}}}

\providecommand{\res}[1]{\Res\displaylimits_{#1}} 

%\providecommand{\norm}[1]{\lVert#1\rVert}
\providecommand{\mtx}[1]{\mathbf{#1}}
\providecommand{\fourier}{\overset{\mathcal{F}}{ \rightleftharpoons}}
%\providecommand{\hilbert}{\overset{\mathcal{H}}{ \rightleftharpoons}}
\providecommand{\system}{\overset{\mathcal{H}}{ \longleftrightarrow}}
	%\newcommand{\solution}[2]{\textbf{Solution:}{#1}}
\newcommand{\solution}{\noindent \textbf{Solution: }}
\newcommand{\cosec}{\,\text{cosec}\,}
\providecommand{\dec}[2]{\ensuremath{\overset{#1}{\underset{#2}{\gtrless}}}}
\newcommand{\myvec}[1]{\ensuremath{\begin{pmatrix}#1\end{pmatrix}}}
\newcommand{\mydet}[1]{\ensuremath{\begin{vmatrix}#1\end{vmatrix}}}
\numberwithin{equation}{subsection}
\makeatletter
\@addtoreset{figure}{problem}
\makeatother
\let\StandardTheFigure\thefigure
\let\vec\mathbf
\renewcommand{\thefigure}{\theproblem}
\def\putbox#1#2#3{\makebox[0in][l]{\makebox[#1][l]{}\raisebox{\baselineskip}[0in][0in]{\raisebox{#2}[0in][0in]{#3}}}}
     \def\rightbox#1{\makebox[0in][r]{#1}}
     \def\centbox#1{\makebox[0in]{#1}}
     \def\topbox#1{\raisebox{-\baselineskip}[0in][0in]{#1}}
     \def\midbox#1{\raisebox{-0.5\baselineskip}[0in][0in]{#1}}
\vspace{3cm}
\title{AI5002 - Assignment 14}
\author{Tuhin Dutta\\ ai21mtech02002}
\maketitle
\newpage
\bigskip
\renewcommand{\thefigure}{\theenumi}
\renewcommand{\thetable}{\theenumi}
\begin{mdframed}
Download code and LaTeX from below hyperlinks\\
1. \href{https://github.com/Tauhait/AI5002/tree/main/Assignment-14/LaTeX}{LaTeX}
\end{mdframed}
\subsection*{\boldsymbol{Problem\ \textbf{NET\_JUNE\_2012\_Q104}}}
Which of the following conditions imply independence of the random variables X
and Y ?\\
\begin{enumerate}
    \item \tag{1} $\pr{X\ \mathop{>}\ a|Y\ \mathop{>}\ a} = \pr{X\ \mathop{>}\ a}\ \forall\ a\ \in\ \mathbb{R}$\\ 
    \item \tag{2} $\pr{X\ \mathop{>}\ a|Y\ \mathop{<}\ b} = \pr{X\ \mathop{>}\ a}\ \forall\ a,\ b\ \in\ \mathbb{R}$\\ 
    \item \tag{3} $X$ and $Y$ are uncorrelated.\\
    \item \tag{4} $E[(X-a)(Y-b)] = E(X-a)\ E(Y-b)\ \forall\ a,\ b \in\ \mathbb{R}$\\
\end{enumerate}
\subsection*{\boldsymbol{Solution}}
We analyze the options one by one and see which option best implies that random variables X and Y are independent. \\
\begin{enumerate}
    \item Let us take a counter example to understand if option 1) being true implies X and Y to be independent random variables.
\begin{align}\tag{1}
    \begin{split}
        &X \sim \mathcal{N}(0, 1)\\
        &Y = 10X\ and\ Y\ \sim \mathcal{N}(0, 100)\\
        &a = -5\\
        &\therefore\ \pr{Y>a} = 1\\
    \end{split}
\end{align}
Now irrespective of the fact that X and Y are dependent, we can write
\begin{align}\tag{2}
    \begin{split}
        \pr{X>a|Y>a} &= \frac{\pr{X>a, Y>a}}{\pr{Y>a}}\\
                     &= \pr{X>a}
    \end{split}
\end{align}
Although the condition in option 1) holds in our example but here X and Y are dependent random variables. Hence option 1) does not imply X and Y to be independent.\\

    \item Let us denote the individual C.D.F.s of the continuous random variables $X$, $Y$ and the joint C.D.F $(X, Y)$, as below,\\
\begin{align}\tag{3}
    \begin{split}
        F_X(a) &= \pr{X \leq a} .= \pr{X < a},\\
        F_Y(b) &= \pr{Y \leq b} = \pr{Y < b}\ and \\
        F_{X,Y}(a,b) &= \pr{X \leq a, Y \leq b}\\
                     &= \pr{X < a, Y < b}
    \end{split}
\end{align}
To show independence, we want to prove that,
\begin{align}\tag{4}
    \begin{split}
        F_X(a)F_Y(b) = F_{X, Y}(a,b)\ \forall\ a,\ b\ \in \mathbb{R}
    \end{split}
\end{align}
From conditional probability we know: 
\begin{align}\tag{5}
    \begin{split}
        \pr{X > a | Y < b} = \frac{\pr{X > a, Y < b}}{\pr{Y < b}}
    \end{split}
\end{align}
and so using the given condition in option 2), we can write (4) as
\begin{align}\tag{6}
    \begin{split}
        \pr{X > a} &= \frac{\pr{X > a, Y < b}}{\pr{Y < b}}\\
        \implies\ \pr{X > a}\pr{Y < b} &= \pr{X > a, Y < b}
    \end{split}
\end{align}
We can write the C.D.F as 
\begin{align}\tag{7}
    \begin{split}
        \pr{X > a} &= 1 - F_X(a)\ and,\\
        \pr{Y < b} &= F_Y(b)
    \end{split}
\end{align}
We may rewrite (5) using (6) as:
\begin{align}\tag{8}
    \begin{split}
        (1 - F_X(a))(F_Y(b)) &= \pr{X > a, Y < b}\\
        F_Y(b) - F_X(a)F_Y(b) &= \pr{X > a, Y < b}\\
        F_X(a)F_Y(b) &= F_Y(b) - \pr{X > a, Y < b}
    \end{split}
\end{align}
Note that since $X$ is continuous so we can write, 
\begin{align}\tag{9}
    \pr{X \leq a} = \pr{X < a}
\end{align}
Regardless of the value of $X$ the marginal C.D.F $F_Y$(b) is given by
\begin{align}\tag{10}
    F_Y(b) = \pr{Y < b}
\end{align}
Now let us define two events 
$$Event\ A: (Y < b \cap X < a)$$ $$Event\ B: (Y < b \cap X > a)$$
We can also think of the event (Y $\mathop{<}$ b) as 
\begin{align}\tag{11}
    (Y < b) = (\text{Event A}) \cup (\text{Event B})
\end{align}
So it implies
\begin{align}\tag{12}
    \pr{A, B} = \pr{Y < b}
\end{align}
Since $X$ cannot both be less than a and greater than a, we have 
\begin{align}\tag{13}
    \begin{split}
        \pr{A, B} &= \pr{A} + \pr{B}\\
                  &= \pr{Y < b, X < a} +\\
                  &\ \ \ \ \pr{Y < b, X > a}
    \end{split}
\end{align}
$\therefore$ We can write
\begin{align}\tag{14}
    \begin{split}
        F_Y(b) = \pr{X > a, Y < b} +\\
                 \pr{X < a, Y < b}
    \end{split}
\end{align}
Now putting value of $F_Y(b)$ from (14) into (8) proves (4) ,
\begin{align}\tag{15}
    \begin{split}
        F_X(a)F_Y(b) &= \pr{X < a, Y < b}\\
                     &= F_{X, Y}(a,b)
    \end{split}
\end{align}\\
Thus (15) shows that the joint C.D.F. is the product of the two individual C.D.F. Hence using the given the condition in option 2) we have proved that X and Y to be independent random variables.\\

    \item Given random variables X and Y are uncorrelated which means that their correlation is 0, or, equivalently, Cov(X, Y) = 0.\\
\begin{align}\tag{16}
    \begin{split}
        Cov(X, Y) &= E\brak{XY} - E\brak{X}E\brak{Y}\\
        &[\because Cov(X, Y) = 0]\\
        E\brak{XY} &= E\brak{X}E\brak{Y}
    \end{split}
\end{align}

We have to prove that uncorrelated implies independence.\\
Let’s take $X$ and $Y$ to exist as an ordered pair at the points (-1, 1), (0, 0), and (1, 1) with probabilities $\dfrac{1}{4}$,$\dfrac{1}{2}$, and $\dfrac{1}{4}$. The expected values of X and Y is\\
\begin{align}\tag{17}
    \begin{split}
        E[X] &= -1\brak{\dfrac{1}{4}} + 0\brak{\dfrac{1}{2}} + 1\brak{\dfrac{1}{4}} = 0 = E[Y]\\
        E[XY] &= -1\brak{\dfrac{1}{4}} + 0\brak{\dfrac{1}{2}} + 1\brak{\dfrac{1}{4}} = 0 = E[X]E[Y]
    \end{split}
\end{align}
Now let's look at the marginal distributions of $X$ and $Y$. $X$ and $Y$ both take on the values $-1,\ 0,\ 1$ and the probability it takes for each of those are given by $\dfrac{1}{4},\ \dfrac{1}{2},\ \dfrac{1}{4}$. Then looping through the possibilities, we have to check if 
\begin{align}\tag{18}
  \pr{X=x, Y=y} = \pr{X=x}\pr{Y=y}  
\end{align}
Let's take the first point (-1, 1) and examine,\\
\begin{align}\tag{19}
    \begin{split}
        &\pr{X=-1, Y=1} = \dfrac{1}{4}\\ &\ne \dfrac{1}{16} = \pr{X=-1}\pr{Y=1}
    \end{split}
\end{align}

We loop through the other two points, and see that X and Y do not meet the definition of independent.\\
Hence it proves that uncorrelated random variables are not always independent. Thus condition in option 3) fail to imply X and Y are independent random variables.\\

    \item We extend L.H.S
\begin{align}\tag{20}
    \begin{split}
        E[(X-a)(Y-b)] = E[XY]\\ - aE[Y] - bE[X] + ab
    \end{split}
\end{align}
and R.H.S to compare,
\begin{align}\tag{21}
    \begin{split}
        E(X-a) E(Y-b) = E[X]E[Y]\\ - aE[Y] - bE[X] + ab
    \end{split}
\end{align}
We see (20) = (21), i.e. independent iff\\
$E[XY] = E[X] E[Y]$ but in this case independence of X and Y cannot be inferred.\\

Let us take a counter example to understand further as to why this condition is not always true.\\
\begin{align}\tag{22}
    \begin{split}
        Let\ &X\ \sim\ \mathcal{N}(0, 1)\\
        &Y = X^2\\
        &\text{Multiply\ X\ on\ both\ sides},\\
        &XY = X^3\\
        \\
        &\text{Let\ us\ calculate\ the\ expected\ values},\\
        &E(XY) = E(X^3) = E(X)E(Y)\\
        &[\text{By\ property\ of\ expectations}]\\
        \\
        &E(XY) = E(X^3) = 0.E(Y) = 0\\
        &[\because\ E(X) = 0]\\
        \\
        \therefore\ &E(XY) = E(X)E(Y)
    \end{split}
\end{align}
From (22) we see they are uncorrelated but not independent because\\
\begin{enumerate}
    \item We have a case when $X \in (0,1)$ and at the same time $Y=X^2>1$. This will never happen, because if $X^2 > 1$ then $X>1$. So the probability is $0$.\\ 
    \item $\pr{0<X<1}\\= F_X(1) - F_X(0)\\ = 0.84134475-0.5\\ = 0.34134475$\\
    \item $P(Y>1)\\=P(X^2>1)\\=P(X>1)+P(X<-1)\\=2*(1-P(X\leq 1))\\=2*(1-F_X(1))\\=2*(1-0.84134475)=0.31731$\\
\end{enumerate}
Hence we can write,
\begin{align}\tag{23}
    \begin{split}
        \pr{0<X<1,Y>1} = 0 \neq\\
        \pr{0<X<1}\pr{Y>1}
    \end{split}
\end{align}
Thus condition in option 4) fail to imply X and Y are independent random variables.
\end{enumerate}
\end{document}
