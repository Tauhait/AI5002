\documentclass[journal,12pt,twocolumn]{IEEEtran}
\usepackage{setspace}
\usepackage{gensymb}
\singlespacing
\usepackage[cmex10]{amsmath}

\usepackage{amsthm}
\usepackage{hyperref}
\hypersetup{
    colorlinks=true,
    linkcolor=blue,
    filecolor=magenta,      
    urlcolor=cyan,
}

\urlstyle{same}
\usepackage{mathrsfs}
\usepackage{txfonts}
\usepackage{stfloats}
\usepackage{bm}
\usepackage{cite}
\usepackage{cases}
\usepackage{subfig}

\usepackage{longtable}
\usepackage{multirow}

\usepackage{enumitem}
\usepackage{mathtools}
\usepackage{steinmetz}
\usepackage{tikz}
\usepackage{circuitikz}
\usepackage{verbatim}
\usepackage{tfrupee}
\usepackage[breaklinks=true]{hyperref}
\usepackage{graphicx}
\usepackage{tkz-euclide}
\usetikzlibrary{shapes,backgrounds}
\usepackage{verbatim}
\usetikzlibrary{calc,math}
\usepackage{listings}
    \usepackage{color}                                            %%
    \usepackage{array}                                            %%
    \usepackage{longtable}                                        %%
    \usepackage{calc}                                             %%
    \usepackage{multirow}                                         %%
    \usepackage{hhline}                                           %%
    \usepackage{ifthen}                                           %%
    \usepackage{lscape}     
\usepackage{multicol}
\usepackage{chngcntr}
\usepackage{mdframed}
\DeclareMathOperator*{\Res}{Res}

\renewcommand\thesection{\arabic{section}}
\renewcommand\thesubsection{\thesection.\arabic{subsection}}
\renewcommand\thesubsubsection{\thesubsection.\arabic{subsubsection}}

\renewcommand\thesectiondis{\arabic{section}}
\renewcommand\thesubsectiondis{\thesectiondis.\arabic{subsection}}
\renewcommand\thesubsubsectiondis{\thesubsectiondis.\arabic{subsubsection}}


\hyphenation{op-tical net-works semi-conduc-tor}
\def\inputGnumericTable{}                                 %%

\lstset{
%language=C,
frame=single, 
breaklines=true,
columns=fullflexible
}

\usepackage{chngcntr}
\counterwithin{figure}{section}

\title{AI5002}
\author{TUHIN DUTTA}
\date{January 2021}

\begin{document}
\newtheorem{theorem}{Theorem}[section]
\newtheorem{problem}{Problem}
\newtheorem{proposition}{Proposition}[section]
\newtheorem{lemma}{Lemma}[section]
\newtheorem{corollary}[theorem]{Corollary}
\newtheorem{example}{Example}[section]
\newtheorem{definition}[problem]{Definition}

\newcommand{\BEQA}{\begin{eqnarray}}
\newcommand{\EEQA}{\end{eqnarray}}
\newcommand{\define}{\stackrel{\triangle}{=}}
\bibliographystyle{IEEEtran}
\raggedbottom
\setlength{\parindent}{0pt}
\providecommand{\mbf}{\mathbf}
\providecommand{\pr}[1]{\ensuremath{\Pr\left(#1\right)}}
\providecommand{\qfunc}[1]{\ensuremath{Q\left(#1\right)}}
\providecommand{\sbrak}[1]{\ensuremath{{}\left[#1\right]}}
\providecommand{\lsbrak}[1]{\ensuremath{{}\left[#1\right.}}
\providecommand{\rsbrak}[1]{\ensuremath{{}\left.#1\right]}}
\providecommand{\brak}[1]{\ensuremath{\left(#1\right)}}
\providecommand{\lbrak}[1]{\ensuremath{\left(#1\right.}}
\providecommand{\rbrak}[1]{\ensuremath{\left.#1\right)}}
\providecommand{\cbrak}[1]{\ensuremath{\left\{#1\right\}}}
\providecommand{\lcbrak}[1]{\ensuremath{\left\{#1\right.}}
\providecommand{\rcbrak}[1]{\ensuremath{\left.#1\right\}}}
\theoremstyle{remark}
\newtheorem{rem}{Remark}
\newcommand{\sgn}{\mathop{\mathrm{sgn}}}

\providecommand{\res}[1]{\Res\displaylimits_{#1}} 

%\providecommand{\norm}[1]{\lVert#1\rVert}
\providecommand{\mtx}[1]{\mathbf{#1}}
\providecommand{\fourier}{\overset{\mathcal{F}}{ \rightleftharpoons}}
%\providecommand{\hilbert}{\overset{\mathcal{H}}{ \rightleftharpoons}}
\providecommand{\system}{\overset{\mathcal{H}}{ \longleftrightarrow}}
	%\newcommand{\solution}[2]{\textbf{Solution:}{#1}}
\newcommand{\solution}{\noindent \textbf{Solution: }}
\newcommand{\cosec}{\,\text{cosec}\,}
\providecommand{\dec}[2]{\ensuremath{\overset{#1}{\underset{#2}{\gtrless}}}}
\newcommand{\myvec}[1]{\ensuremath{\begin{pmatrix}#1\end{pmatrix}}}
\newcommand{\mydet}[1]{\ensuremath{\begin{vmatrix}#1\end{vmatrix}}}
\numberwithin{equation}{subsection}
\makeatletter
\@addtoreset{figure}{problem}
\makeatother
\let\StandardTheFigure\thefigure
\let\vec\mathbf
\renewcommand{\thefigure}{\theproblem}
\def\putbox#1#2#3{\makebox[0in][l]{\makebox[#1][l]{}\raisebox{\baselineskip}[0in][0in]{\raisebox{#2}[0in][0in]{#3}}}}
     \def\rightbox#1{\makebox[0in][r]{#1}}
     \def\centbox#1{\makebox[0in]{#1}}
     \def\topbox#1{\raisebox{-\baselineskip}[0in][0in]{#1}}
     \def\midbox#1{\raisebox{-0.5\baselineskip}[0in][0in]{#1}}
\vspace{3cm}
\title{AI5002 - Assignment 14}
\author{Tuhin Dutta\\ ai21mtech02002}
\maketitle
\newpage
\bigskip
\renewcommand{\thefigure}{\theenumi}
\renewcommand{\thetable}{\theenumi}
\begin{mdframed}
Download code and LaTeX from below hyperlinks\\
1. \href{https://github.com/Tauhait/AI5002/tree/main/Assignment-14/LaTeX}{LaTeX}
\end{mdframed}
\subsection*{\boldsymbol{Problem\ \textbf{NET\_JUNE\_2012\_Q104}}}
Which of the following conditions imply independence of the random variables X
and Y ?\\
\begin{enumerate}
    \item \tag{1} p(X $\mathop{>}$ a $|$ Y $\mathop{>}$ a) = P(X $\mathop{>}$ a) for all a $\in\ \mathbb{R}$\\ 
    \item \tag{2} p(X $\mathop{>}$ a $|$ Y $\mathop{<}$ b) = P(X $\mathop{>}$ a) for all a, b $\in\ \mathbb{R}$\\
    \item \tag{3} X and Y are uncorrelated.\\
    \item \tag{4} $E[(X-a)(Y-b)] = E(X-a)\ E(Y-b)\ \forall\ a,\ b \in\ \mathbb{R}$\\
\end{enumerate}
\subsection*{\boldsymbol{Solution}}
We analyze the options one by one and see which option best implies that random variables X and Y are independent. \\
\begin{enumerate}
    \item In case of option (1)\\
Let's assume continuous r.v.s X and Y are not independent and, \\
\begin{align}\tag{1}
    \begin{split}
        X \in \{0, 1\}\\
        Y = X + 2\\
        \pr{X=0} = \pr{X=1} = \dfrac{1}{2}\\
    \end{split}
\end{align}
Now since Y is always greater than X therefore $\pr{X>a \;| Y>a}$ equals $\pr{X > a}\ \forall\ a\ \in \mathbb{R}$ and thus in spite of the option (1) being true, it fails to imply that X and Y are independent random variables and hence option (1) is false.\\
    \item In case of option (2),\\
Let us denote the cumulative distribution functions of $X$, $Y$ and $(X, Y)$, as below,\\
\begin{align}\tag{2}
    \begin{split}
        F_X(a) &= P(X \leq a),\\
        F_Y(b) &= P(Y \leq b)\ and \\
        F_{X,Y}(a,b) &= P(X \leq a \mbox{ and } Y \leq b) 
    \end{split}
\end{align}
To show independence, we want to prove that,
\begin{align}\tag{3}
    \begin{split}
        F_X(a)F_Y(b) = F_{X, Y}(a,b)\ \forall\ a,\ b\ \in \mathbb{R}
    \end{split}
\end{align}
Conditional probability tells us that: 
\begin{align}\tag{4}
    \begin{split}
        P(X > a | Y < b) = \frac{P(X > a \mbox{ and } Y < b)}{P(Y < b)}
    \end{split}
\end{align}
and so by the assumptions of the option (1),
\begin{align}\tag{5}
    \begin{split}
        P(X > a) &= \frac{P(X > a \mbox{ and } Y < b)}{P(Y < b)}\\
        P(X > a)P(Y < b) &= P(X > a \mbox{ and } Y < b)
    \end{split}
\end{align}
Now since 
\begin{align}\tag{6}
    \begin{split}
        P(X > a) &= 1 - F_X(a)\ and,\\
        P(Y < b) &= F_Y(b)
    \end{split}
\end{align}
we may rewrite the above equation as:
\begin{align}\tag{7}
    \begin{split}
        F_Y(b) - F_X(a)F_Y(b) &= P(X > a \mbox{ and } Y < b)\\
        F_X(a)F_Y(b) &= F_Y(b) - P(X > a \mbox{ and } Y < b)
    \end{split}
\end{align}
Also, note that 
\begin{align}\tag{8}
    \begin{split}
        F_Y(b) = P(X > a \mbox{ and } Y < b) + P(X < a \mbox{ and } Y < b)
    \end{split}
\end{align}
Thus putting value of $F_Y(b)$ from (8) into (7) proves (2) ,
\begin{align}\tag{9}
    \begin{split}
        F_X(a)F_Y(b) &= P(X < a \mbox{ and } Y < b)\\
    \end{split}
\end{align}\\
Thus option (2) seems to be always true.\\
    \item In case of option (3),\\
Given random variables X and Y are uncorrelated which means that their correlation is 0, or, equivalently, Cov(X, Y) = 0.\\
\begin{align}\tag{10}
    \begin{split}
        Cov(X, Y) &= E[XY] - E[X]E[Y]\\
        &[\because Cov(X, Y) = 0]\\
        E[XY] &= E[X]E[Y]
    \end{split}
\end{align}

We have to prove that uncorrelated implies independence.\\
Let’s take $X$ and $Y$ to exist as an ordered pair at the points (-1,1), (0,0), and (1,1) with probabilities $\dfrac{1}{4}$,$\dfrac{1}{2}$, and $\dfrac{1}{4}$. The expected values of X and Y is\\
\begin{align}\tag{11}
    \begin{split}
        E[X] = -1\cdot\dfrac{1}{4} + 0\cdot\dfrac{1}{2} + 1\cdot\dfrac{1}{4} = 0 = E[Y]\\
        E[XY] = -1\cdot\dfrac{1}{4} + 0\cdot\dfrac{1}{2} + 1\cdot\dfrac{1}{4} = 0 = E[X]E[Y]
    \end{split}
\end{align}\\

Now let's look at the marginal distributions of $X$ and $Y$. $X$ and $Y$ both take on the values $-1,\ 0,\ 1$ and the probability it takes for each of those are given by $\dfrac{1}{4},\ \dfrac{1}{2},\ \dfrac{1}{4}$. Then looping through the possibilities, we have to check if $$P(X=x, Y=y) = P(X=x) P(Y=y)$$\\
Let's take the first point (-1, 1) and examine,\\
\begin{align}\tag{12}
    \begin{split}
        P(X=-1, Y=1) = \dfrac{1}{4} \ne \dfrac{1}{16} = P(X=-1)\ P(Y=1)
    \end{split}
\end{align}
We loop through the other two points, and see that X and Y do not meet the definition of independent.\\
Thus it proves that uncorrelated random variables are not always independent. Hence option (3) is false.\\
    \item In case of option (4),we have
\begin{align}\tag{13}
    E[(X-a)(Y-b)] &= E[XY] - aE[Y] - bE[X] + ab
\end{align}
Also,
\begin{align}\tag{14}
    E(X-a) E(Y-b) &= E[X]E[Y] - aE[Y] - bE[X] + ab
\end{align}
We see (13) = (14), i.e. independent iff\\
$E[XY] = E[X] E[Y]$ but in this case independence of X and Y cannot be inferred. Hence option (4) is false.
\end{enumerate}
\end{document}
