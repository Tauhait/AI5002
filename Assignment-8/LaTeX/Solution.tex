\documentclass[journal,12pt,twocolumn]{IEEEtran}
\usepackage{setspace}
\usepackage{gensymb}
\singlespacing
\usepackage[cmex10]{amsmath}

\usepackage{amsthm}
\usepackage{hyperref}
\hypersetup{
    colorlinks=true,
    linkcolor=blue,
    filecolor=magenta,      
    urlcolor=cyan,
}

\urlstyle{same}
\usepackage{placeins}
\usepackage{mathrsfs}
\usepackage{txfonts}
\usepackage{stfloats}
\usepackage{bm}
\usepackage{cite}
\usepackage{cases}
\usepackage{subfig}
\usepackage{amsmath}
\usepackage{longtable}
\usepackage{multirow}
\usepackage{float}
\restylefloat{table}
\usepackage{enumitem}
\usepackage{mathtools}
\usepackage{steinmetz}
\usepackage{tikz}
\usepackage{circuitikz}
\usepackage{verbatim}
\usepackage{tfrupee}
\usepackage[breaklinks=true]{hyperref}
\usepackage{graphicx}
\usepackage{tkz-euclide}
\usetikzlibrary{shapes,backgrounds}
\usepackage{verbatim}
\usetikzlibrary{calc,math}
\usepackage{listings}
    \usepackage{color}                                            %%
    \usepackage{array}                                            %%
    \usepackage{longtable}                                        %%
    \usepackage{calc}                                             %%
    \usepackage{multirow}                                         %%
    \usepackage{hhline}                                           %%
    \usepackage{ifthen}                                           %%
    \usepackage{lscape}     
\usepackage{multicol}
\usepackage{chngcntr}
\usepackage{mdframed}
\DeclareMathOperator*{\Res}{Res}

\renewcommand\thesection{\arabic{section}}
\renewcommand\thesubsection{\thesection.\arabic{subsection}}
\renewcommand\thesubsubsection{\thesubsection.\arabic{subsubsection}}

\renewcommand\thesectiondis{\arabic{section}}
\renewcommand\thesubsectiondis{\thesectiondis.\arabic{subsection}}
\renewcommand\thesubsubsectiondis{\thesubsectiondis.\arabic{subsubsection}}


\hyphenation{op-tical net-works semi-conduc-tor}
\def\inputGnumericTable{}                                 %%

\lstset{
%language=C,
frame=single, 
breaklines=true,
columns=fullflexible
}

\usepackage{chngcntr}
\counterwithin{figure}{section}

\title{AI5002}
\author{TUHIN DUTTA}
\date{January 2021}

\begin{document}
\newtheorem{theorem}{Theorem}[section]
\newtheorem{problem}{Problem}
\newtheorem{proposition}{Proposition}[section]
\newtheorem{lemma}{Lemma}[section]
\newtheorem{corollary}[theorem]{Corollary}
\newtheorem{example}{Example}[section]
\newtheorem{definition}[problem]{Definition}

\newcommand{\BEQA}{\begin{eqnarray}}
\newcommand{\EEQA}{\end{eqnarray}}
\newcommand{\define}{\stackrel{\triangle}{=}}
\bibliographystyle{IEEEtran}
\raggedbottom
\setlength{\parindent}{0pt}
\providecommand{\mbf}{\mathbf}
\providecommand{\pr}[1]{\ensuremath{\Pr\left(#1\right)}}
\providecommand{\qfunc}[1]{\ensuremath{Q\left(#1\right)}}
\providecommand{\sbrak}[1]{\ensuremath{{}\left[#1\right]}}
\providecommand{\lsbrak}[1]{\ensuremath{{}\left[#1\right.}}
\providecommand{\rsbrak}[1]{\ensuremath{{}\left.#1\right]}}
\providecommand{\brak}[1]{\ensuremath{\left(#1\right)}}
\providecommand{\lbrak}[1]{\ensuremath{\left(#1\right.}}
\providecommand{\rbrak}[1]{\ensuremath{\left.#1\right)}}
\providecommand{\cbrak}[1]{\ensuremath{\left\{#1\right\}}}
\providecommand{\lcbrak}[1]{\ensuremath{\left\{#1\right.}}
\providecommand{\rcbrak}[1]{\ensuremath{\left.#1\right\}}}
\theoremstyle{remark}
\newtheorem{rem}{Remark}
\newcommand{\sgn}{\mathop{\mathrm{sgn}}}

\providecommand{\res}[1]{\Res\displaylimits_{#1}} 

%\providecommand{\norm}[1]{\lVert#1\rVert}
\providecommand{\mtx}[1]{\mathbf{#1}}
\providecommand{\fourier}{\overset{\mathcal{F}}{ \rightleftharpoons}}
%\providecommand{\hilbert}{\overset{\mathcal{H}}{ \rightleftharpoons}}
\providecommand{\system}{\overset{\mathcal{H}}{ \longleftrightarrow}}
	%\newcommand{\solution}[2]{\textbf{Solution:}{#1}}
\newcommand{\solution}{\noindent \textbf{Solution: }}
\newcommand{\cosec}{\,\text{cosec}\,}
\providecommand{\dec}[2]{\ensuremath{\overset{#1}{\underset{#2}{\gtrless}}}}
\newcommand{\myvec}[1]{\ensuremath{\begin{pmatrix}#1\end{pmatrix}}}
\newcommand{\mydet}[1]{\ensuremath{\begin{vmatrix}#1\end{vmatrix}}}
\numberwithin{equation}{subsection}
\makeatletter
\@addtoreset{figure}{problem}
\makeatother
\let\StandardTheFigure\thefigure
\let\vec\mathbf
\renewcommand{\thefigure}{\theproblem}
\def\putbox#1#2#3{\makebox[0in][l]{\makebox[#1][l]{}\raisebox{\baselineskip}[0in][0in]{\raisebox{#2}[0in][0in]{#3}}}}
     \def\rightbox#1{\makebox[0in][r]{#1}}
     \def\centbox#1{\makebox[0in]{#1}}
     \def\topbox#1{\raisebox{-\baselineskip}[0in][0in]{#1}}
     \def\midbox#1{\raisebox{-0.5\baselineskip}[0in][0in]{#1}}
\vspace{3cm}
\title{AI5002 - Assignment 8}
\author{Tuhin Dutta\\ ai21mtech02002}
\maketitle
\newpage
\bigskip
\renewcommand{\thefigure}{\theenumi}
\renewcommand{\thetable}{\theenumi}
\begin{mdframed}
Download code and LaTeX from below hyperlinks\\
1. \href{https://github.com/Tauhait/AI5002/blob/main/Assignment-8/Codes/MiscDist\_5\_15.py}{Codes/MiscDist\_5\_15.py}


2. \href{https://github.com/Tauhait/AI5002/tree/main/Assignment-8/LaTeX}{LaTeX}
\end{mdframed}
\subsection*{\boldsymbol{Problem\ 5.15}}
\begin{flushleft} State which of the following are not the
probability distributions of a random variable.
Give reasons for your answer.\end{flushleft}
\newline i)
\begin{table}[ht]
\begin{tabular}{|l|l|l|l|}
\hline
\textbf{X}    & 0   & 1   & 2   \\ \hline
\textbf{P(X)} & 0.4 & 0.4 & 0.2 \\ \hline
\end{tabular}
\end{table}
\newline ii)
\begin{table}[ht]
\begin{tabular}{|l|l|l|l|l|l|}
\hline
\textbf{X}    & 0   & 1   & 2   & 3    & 4   \\ \hline
\textbf{P(X)} & 0.1 & 0.5 & 0.2 & -0.1 & 0.3 \\ \hline
\end{tabular}
\end{table}
\newline iii)
\begin{table}[ht]
\begin{tabular}{|l|l|l|l|}
\hline
\textbf{X}    & -1   & 0   & 1   \\ \hline
\textbf{P(X)} & 0.6 & 0.1 & 0.2 \\ \hline
\end{tabular}
\end{table}
\newline iv)
\begin{table}[ht]
\begin{tabular}{|l|l|l|l|l|l|}
\hline
\textbf{X}    & 3   & 2   & 1   & 0    & -1   \\ \hline
\textbf{P(X)} & 0.3 & 0.2 & 0.4 & 0.1 & 0.05 \\ \hline
\end{tabular}
\end{table}
\subsection*{\boldsymbol{Solution}}\\
\begin{flushleft}
Consider an experiment whose sample space is S. For each event E of the sample
space S, we assume that a number P(E) is defined and satisfies the following three
axioms -
\newline
We refer to P(E) as the probability of the event E.
\newline
\newline
\newline
\newline
\newline
\text{Axiom 1:}
\begin{align}
    0 \leq P (E) \leq 1
\end{align}
It states that the probability that the outcome of the experiment is an outcome in E is some number between 0 and 1.
\newline
\newline
\text{Axiom 2:}
\begin{align}
    P (S) = 1
\end{align}
It states that, with probability 1, the outcome will be a point in the sample space S.
\newline
\newline
\text{Axiom 3:}
\begin{align}
    P\ \Bigg(\bigcup_{i=1}^{\infty} \mathop{E_i} \Bigg) = \sum_{i=1}^{\infty} P (\mathop{E_i})
\end{align}
It states that, for any sequence of mutually exclusive events, the probability of at least one of these events occurring is just the sum of their respective probabilities.
\newline
\newline
From figure i) we see that, the sum of probabilities is = 0.4 + 0.4 + 0.2 = 1.
\newline Thus X is a valid probability distribution.
\newline
\newline
From figure ii) we see that, a negative probability -0.1 is given which is impossible.
\newline Thus X is NOT a valid probability distribution.
\newline
\newline
From figure iii) we see that, the sum of probabilities is = 0.6 + 0.1 + 0.2 = 0.9 $\neq$ 1 .
\newline Thus X is NOT a valid probability distribution.
\newline
\newline
From figure iv) we see that, the sum of probabilities is = 0.3 + 0.2 + 0.4 + 0.1 + 0.05 = 1.05 $\neq$ 1.
\newline Thus X is NOT a valid probability distribution.
\newline
\end{flushleft}

\end{document}
